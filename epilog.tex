\chapter*{Conclusion}
\addcontentsline{toc}{chapter}{Conclusion}

We have succeeded in implementing the multicast routing features to the BIRD.
Because the protocol is fairly complex, it deserves to be more tested and, what
is more important, reviewed by the BIRD team. They were too busy to deep-study
my implementation continuously. But merging the PIM into BIRD is on the plan.

The source code for the whole project is available on the
Github\footnote{https://github.com/Aearsis/bird}. As it is versioned by the
git, you can clearly see what my contributions are.

There were a considerable amount of obstacles during the work. First, I had to
learn how the BIRD itself works. It is not a small project, and there are lots
of internal details one must know before starting to contribute in a major way.
Also, I have decided to change the existing BIRD code as little as possible and
write my code in the spirit of the existing one, to follow the coding practises
and used design patterns. That made me study the code for even longer time, but
hopefully it will pay off when merging to the main development branch.

Next, there were problems with undocumented code in Linux kernel. Looking back,
when we know where to look for the issues, they seem trivial. But they blocked
the work for a while. Namely, all the surprises around the IGMP control socket,
details about how MFC entries work or the Linux bridge doing IGMP snooping by
default.

Naturally, there are many ways how to expand this project. The first dorection
should be a support for IPv6. It should be fairly simple, because the internal
structures are prepared for it. Also, the PIM itselg should be able to run on
IPv6 out of the box, but this behavior is not tested. One just need to
implement the MLD protocol (IPv6 equivalent of the IGMP), and the IPv6 kernel
route synchronization. In the kernel, the v6 API looks like a copy of the v4,
on which a few text substitutions were applied.

Next, it would be great to implement another modes of the PIM, namely PIM-SM.
Again, the design of the PIM protocol is prepared for it. One would need to
add the different types of packets PIM-SM uses and prepend source address to
the internal structures. It will probably not be that easy, but we cannot see
any fundamental problem now.

For the bravest, the part that needs reimplementation the most is the Linux
kernel itself. The implementation of multicast forwarding is copied from BSD to
stay compatible with the mrouted daemon, and it reflects how mrouted works.
That is not always the most convenient way. Also, the multicast input, IGMP
processing and other related chunks of code looks more like a hack than
a systematical solution. It would be great to have a clean netlink API to
control the multicast routing table, without the need to control everything
through one almighty socket.

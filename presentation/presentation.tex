\documentclass{beamer}
\usepackage[utf8]{inputenc}
\usepackage[czech]{babel}
\usepackage{fancyvrb,ulem}
\usetheme{Madrid}
\usecolortheme{seahorse}

\title{Multicast Routing}
\subtitle{Směrování multicastu}
\author{Ondřej Hlavatý}
\institute{MFF UK}

\newcommand{\vrbalert}[2]{\alert<#1>{#2}}

\begin{document}

\frame{\titlepage}

\begin{frame}\frametitle{Cíl práce}
  \begin{center}
    \huge
    Rozšíření směrovacího démonu BIRD \\
    o schopnost směrování multicastu.
  \end{center}
\end{frame}

\begin{frame}\frametitle{BIRD}
  \begin{itemize}
    \item UNIX, C
    \item Původem softwarový projekt na MFF (1998)
      \vfill {\small\it
    \item Démon = tradiční název pro službu běžící na pozadí}
  \end{itemize}
\end{frame}


\begin{frame}[t,fragile]\frametitle{Možnosti posílání zpráv}
	\vbox to 0pt{%
	\hfill\includegraphics<1>[scale=0.7]{img/cast_unicast.pdf}%
	\vss}\nointerlineskip\vbox to 0pt{%
	\hfill\includegraphics<2>[scale=0.7]{img/cast_broadcast.pdf}%
	\vss}\nointerlineskip\vbox to 0pt{%
	\hfill\includegraphics<3>[scale=0.7]{img/cast_multicast.pdf}%
	\vss}\nointerlineskip %
	\begin{itemize}[<+->]
	  \item Unicast
	  \item Broadcast
	  \item Multicast
	  \begin{itemize}[<.->]
	    \item \texttt{224.0.0.0/4}
	    \item \texttt{IGMP}
	  \end{itemize}
	\end{itemize}
\end{frame}

\begin{frame}[fragile]
	\frametitle{Směrování unicastu}

	\vbox to 72pt{
	\hfill\includegraphics[width=0.5\textwidth]{img/unicast_source_alfa.pdf}
	\vss}

\begin{Verbatim}[commandchars=\\\{\}]
# ip route show
\vrbalert{2}{alfa  dev eth-alfa proto kernel metric 1}
\vrbalert{2}{B     dev ptp-b    proto kernel metric 1}
\vrbalert{3}{C     via B        proto bird   metric 2}
\vrbalert{3}{beta  via B        proto bird   metric 2}
\vrbalert{3}{D     via B        proto bird   metric 2}
\vrbalert{3}{gamma via B        proto bird   metric 3}
\vrbalert{3}{delta via B        proto bird   metric 3}
\vrbalert{2}{C     dev ptp-c    proto kernel metric 100}
\end{Verbatim}
\end{frame}

\begin{frame}
	\frametitle{Směrování unicastu -- distribuční strom cíle}
	
	\begin{center}
	\includegraphics[width=0.8\textwidth]{img/unicast_dest_delta.pdf}
	\end{center}
\end{frame}

\begin{frame}[fragile]
  \frametitle{Směrování multicastu}

	\begin{itemize}
	  \item Multicast Forwarding Cache
	\end{itemize}
\begin{verbatim}
# ip mroute show
(10.0.0.1, 224.42.0.1)   Iif: br0   Oifs: wlan0 eth0
(10.0.0.2, 224.42.0.2)   Iif: br0   Oifs: eth0
(10.0.1.2, 224.42.0.3)   Iif: eth0  Oifs:
\end{verbatim}
	\begin{itemize}
	  \item Pro zprávu od odesílatele a cílovou skupinu: \\
	    \quad pokud zpráva přišla z \texttt{Iif}, rozešli na všechny Oifs
	  \item Nelze nastavovat \uv{ručně}
	  \item Navíc udržování příjemců
	\end{itemize}
\end{frame}

\begin{frame}\frametitle{Protocol Independent Multicast -- Bidirectional}
  \begin{center}
    \includegraphics[width=\textwidth]{img/rp-tree-bidir.pdf}
  \end{center}
\end{frame}

\begin{frame}\frametitle{Protocol Independent Multicast -- Bidirectional}
  \begin{itemize}
    \item RFC 5015 (proposed standard)
    \item Rendezvous-point (RP) společný pro skupiny
    \item Jmenovaný směrovač pro síť (Designated Forwarder, DF) \\
      \bigskip \pause
      $\rightarrow$ unicastový distribuční strom cíle pro $RP$
  \end{itemize}
\end{frame}

\begin{frame}[plain,c]
  \begin{center}
    \usebeamerfont*{frametitle}
    \Huge Implementace
  \end{center}
\end{frame}

\begin{frame}[fragile]\frametitle{BIRD}
  \vbox to 0pt{\vss{\hfill\includegraphics{img/bird.pdf}\qquad\strut}\vss}\nointerlineskip\vbox to 0pt{\vss
  \begin{itemize}
    \item Tabulky
    \item Protokoly
    \item<3> Filtry
  \end{itemize}\vss}
\end{frame}

\begin{frame}\frametitle{Návrh řešení směrování multicastu}
  \begin{itemize}
    \item Zobecnění routovacích tabulek pro multicastové routy a požadavky
    \item Implementace třech BIRD protokolů: IGMP, PIM a rozhraní k jádru
  \end{itemize}
  \bigskip
  \begin{center}
    \includegraphics<.->[scale=0.8]{img/channels-pres.pdf}
  \end{center}
\end{frame}

\begin{frame}\frametitle{IGMP}
  \begin{itemize}
    \item Komunikuje s cílovými stanicemi
    \item Udržuje \uv{multicastové požadavky} \\
      $\quad=$ na rozhraní $I$ je někdo ve skupině $G$
    \item Musí přijímat všechny IGMP zprávy, tedy obcházet filtry v jádře \\
      $\quad\rightarrow$ speciální socket
  \end{itemize}
\end{frame}

\begin{frame}\frametitle{mkernel}
  \begin{itemize}
    \item Vyhodnocuje výpadky MFC, komunikuje s jádrem
    \item API je původní z BSD (\texttt{setsockopt}, \texttt{ioctl})
  \end{itemize}
\end{frame}

\begin{frame}\frametitle{PIM}
  \begin{itemize}
    \item Rozdělen na tři logické části: \par
      \medskip
      \begin{center}
      \includegraphics[scale=0.8]{img/pim-pres.pdf}
      \end{center}
    \item \texttt{DF} -- udržování RP stromu
    \item \texttt{JP} -- udržování multicastových požadavků
    \item \texttt{FW} -- logika za přesměrováváním zpráv
  \end{itemize}
\end{frame}

\begin{frame}\frametitle{Testování}
  \begin{itemize}
    \item Distribuovaný výpočet
    \item Až na komplexní síti se projeví funkce
    \item Testovací framework \texttt{netlab}
  \end{itemize}
\end{frame}

\begin{frame}[fragile]\frametitle{Netlab}
   \vbox to 0pt{\vskip 11pt\hfill\includegraphics{img/pres-netlab.pdf}\qquad\strut\vss}\nointerlineskip
  \begin{itemize}
    \item Deklarativní popis sítě
\begin{verbatim}
	  netlab_node r1
	  netlab_node r2
	  if_veth r1 r2 10.10.1
\end{verbatim}

    \item Jednoduché CLI vycházející z potřeb testování

\begin{verbatim}
	  # ./netlab
	  [netlab] # net_up
	  [netlab] # start
	  [netlab] # log r1
	  [netlab] # restart r1
	  [netlab] # stop
	  [netlab] # net_down
\end{verbatim}

  \end{itemize}
\end{frame}

\begin{frame}\frametitle{Ukázka}
\includegraphics[width=\textwidth]{img/netlab.pdf}
\end{frame}

\begin{frame}\frametitle{Budoucnost}
  \begin{itemize}
    \item Integrace do vývojové větve BIRDa
    \item Rozšíření na IPv6
    \item Rozšíření o další režimy PIM
    \item Refaktorizace jádra
  \end{itemize}
\end{frame}

\begin{frame}[plain]
  \begin{center}
    Děkuji za pozornost.
  \end{center}
\end{frame}
\end{document}

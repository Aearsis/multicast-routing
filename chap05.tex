\chapter{Multicast in the BIRD -- configuration}

The multicast routing features are part of the BIRD, and as such they are
configurable through the BIRD configuration mechanism. We also tried to make
the configuration fit well into the existing scheme. Configuration option
names are chosen to reflect the variable names used in the standards.

To use the multicast routing features in the BIRD, you must enable at least two
protocols: PIM and mkernel. When any of the router's links have hosts
connected, you probably want to enable also the IGMP. Though you have to
include these protocols in your configuration, all the protocol-specific
options have reasonable default values suggested by the protocols standards, and
you do not need to include them at all. This chapter covers the complete
configuration of all three protocols.

\section{The IGMP}

The IGMP listens for hosts' multicast reports, and fills the request table in
the BIRD. The default values are well suited for the majority of use cases, but
you still need to include the IGMP configuraton section to make IGMP run. Do it
so, whenever a router is connected to a link with local hosts. So, the most
common configuration will be the following:

\begin{lstlisting}
protocol igmp { }
\end{lstlisting}

\noindent All the configurable variables have the same names as in the IGMP standard, \rfc{2236}.
Intervals are specified by time expressions to avoid unit problems. Have a~look
at a configuration example:

\begin{lstlisting}
protocol igmp {
  interface "eth*" {
    robustness 2;
    query interval 125 s;
    query response interval 10 s;
    startup query count 2;
    startup query interval 31 s;
    last member query count 2;
    last member query interval 1 s;
  };
  mreq4 { import all; };
}
\end{lstlisting}

\noindent All configuration variables are interface-specific. Let us go through their meaning:

\newcommand{\cfg}[3]{\hbox to \hsize{\ttt{#1} $#2$ \rm (#3)\hfill}}
\newcommand{\cfgitem}[1]{\item[\hbox to \hsize{\vbox{#1}}]}
\begin{description}[style=nextline]
\cfgitem{\cfg{robustness}{r}{integer}}
  Loss of less than $r$ packets cannot cause the protocol to have inconsistent
  states. Increasing $r$ will increase stability on lossy links, but groups
  will be left more slowly.

  Default value: 2

\cfgitem{\cfg{query interval}{t}{time}}
  A general query will be sent once per $t$. Decreasing $t$ will cause more
  control traffic, but faster convergence on lossy links.

  Default value: 125 seconds

\cfgitem{\cfg{query response interval}{t}{time}}
  Hosts have to send their report until $t$ passes. Larger values spread the
  burst of control traffic after a general query.

  Default value: 10 seconds

\cfgitem{
  \cfg{startup query count}{c}{integer}
  \cfg{startup query interval}{t}{time}
}
  When starting up, a loss of a query is critical, because it causes undetected
  delay before starting to forward traffic. First $c$ queries are sent in
  shorter interval $t$.

  Default values: \ttt{robustness} messages, $\frac14$ the \ttt{query interval}

\cfgitem{
  \cfg{last member query count}{c}{integer}
  \cfg{last member query interval}{t}{time}
}
  After receiving a leave, $c$ group-specific queries are sent to the group,
  once per $t$. After time $c\cdot t$, the router assumes there are no group members.
  Decreasing these will cause groups to be left faster.

  Default values: \ttt{robustness} messages, 1 second

\end{description}

\noindent There is also an implicit channel configuration for the multicast request
table. An IGMP instance can have only one channel of type \texttt{mreq4}
configured. The defaults filter settings on a channel are well suited here, so
you can omit the the channel configuration completely. The IGMP rejects all
route updates.

\section{The PIM protocol}

PIM tries to satisfy the multicast requests by joining the interface into
a distribution tree. To use PIM to manage a group, you must configure its RP
address. You can specify a common RPA for a range of groups using a CIDR
notation:
\begin{lstlisting}
protocol pim {
  group 224.42.0.0/16 {
    rpa 10.0.0.1;
  };
}
\end{lstlisting}

\noindent In order to function properly, all routers must be configured to use the same
RPA for a group. When choosing a RPA, bear in mind that every multicast packet
will have to be transfered to the link where RPA belongs. If you have only one
fixed source, the best choice for RPA is the source address. Otherwise you can
choose some address belonging to the backbone of your network.

Other options for PIM are for fine-tuning the constants for specific
interfaces (slow or lossy). An example follows:

\begin{lstlisting}
protocol pim {
  interface "*" {
    hello period 30 s;
    hello delay 5 s;
    hello holdtime 105 s;

    election robustness 3;

    override interval 3 s;
    joinprune period 60 s;
    joinprune holdtime 60 s;
  };
}
\end{lstlisting}

\begin{description}[style=nextline]
\cfgitem{\cfg{hello period}{t}{time}}
  A PIM hello packet is sent once per $t$. PIM does not use BFD, and it depends
  on hello packets to detect a neighbor failure.

  The default value is 30 seconds.

\cfgitem{\cfg{hello holdtime}{t}{time}}
  This value is announced to neighbors. After a period of $t$ without receiving
  a hello packet from this router, neighbors assume this router is dead.

  If omitted, its value is calculated as $3.5\mathop{\cdot}\hbox{hello period}$.

\cfgitem{\cfg{hello delay}{t}{time}}
  A maximum value for the random delay before sending a first hello. It
  prevents bursts of control traffic when the link becomes up.

  The default value is 5 seconds.

\cfgitem{\cfg{election robustness}{r}{integer}}
  The minimum number of packets that must be lost to break the DF election
  mechanism. Every message is transmitted $r$ times. The larger the $r$, the
  slower the election will be, but it will be more prone to packet loss.

  The default value is 3.

  The delays between election messages are not configurable, their values are
  specified in the standard.

\cfgitem{\cfg{override interval}{t}{time}}
  When a prune is received, other routers on a link interested in a traffic for
  a~group $G$ must send their overriding join during $t$. After a delay of $t$, this
  router assumes there are no other routers it and will stop forwarding $G$ to this
  link.

  Note that this value can be changed in the runtime through options in hello
  messages.

  The default is 3 seconds.

\cfgitem{\cfg{joinprune period}{t}{time}}
  A Join message for every group will be sent once per $t$, unless an other router
  is sending them.

  The default value is 60 seconds.

\cfgitem{\cfg{joinprune holdtime}{t}{time}}
  This value is transmitted to the upstream neighbor. After a period of $t$,
  the neighbor assumes we have pruned the group, but the Prune packet was
  lost.

  If not present, the value is set to $3.5\mathop{\cdot}\ttt{joinprune period}$.
\end{description}

The channel configuration of PIM is more complicated than of other protocols,
because PIM is connected to three routing tables of different types. The basic
setting (not the default though):

\begin{lstlisting}
protocol pim {
  ipv4  { import none; export all;  };
  mreq4 { import all;  export all;  };
  mgrp4 { import all;  export none; };
}
\end{lstlisting}

\begin{figure}[htp]
\centering
\includegraphics[scale=1.00]{img/channels.pdf}
\caption{A scheme of channels connecting routing tables with protocols}
\label{channels}
\end{figure}

\noindent The multicast route channel (\ttt{mgrp4}) has the default setting, causing
multicast routes to be propagated into kernel, and no routes to be announced
back. It does not make much sense to learn multicast routes from other
protocols.

If you leave the default \ttt{export none} setting for the multicast request
channel \ttt{mreq4}, PIM will only forward traffic upstream. Reasons for the
decision to loop PIM internal requests through the multicast requests table can
be found in section~\ref{why-pim-mreq}.

The last channel, \ttt{ipv4}, is providing the network topology, i.e. the MRIB,
to the routing protocol. The topology is read-only for PIM, no routes are
imported back to the routing table. If you do not supply any topology to PIM
(for example by leaving the export filter on its default value \ttt{none}), PIM
will not know its upstream interface and therefore cannot join the shared
tree. You can see the scheme of all the channels on the figure \ref{channels}.

By changing the MRIB channel, you can unleash the full power of BIRD. To give
an example, imagine you are running a system of networks running BGP. As
a~metric, you are using the link reliability, having the BIRD select the most
reliable route as the best one. Now you would like to use multicast for
videoconferences. For transfering video, there are more important properties of
the connection than reliability -- bandwidth, low and stable latency. In BIRD,
this is very easy. Just create an internal routing table, run another routing
protocol (either a different one, or BGP with different settings) and supply
its topology to PIM as the MRIB. Unicast is still routed the same, but
multicasts will use a~different path.

When the PIM is electing the designated forwarder, it advertises the IGP metric, and
the one with the best metric becomes the forwarder. You can alter the metric
exported to PIM to any value you want, so you do not even have to run two
separate routing protocols.
\nomenclature{IGP}{Interior Gateway Protocol}

\section{The mkernel protocol}

There is nothing you can configure in the mkernel protocol directly. You only
need to include the configuration section to enable the multicast routing
features.

\begin{lstlisting}
protocol mkernel {
  mgrp4 { export all; };
}
\end{lstlisting}

\noindent You must still configure the only channel mkernel uses. By default, it
is connected to the routing table \ttt{mroute4}. As mkernel never announces any
route, the default filter settings are pretty much useless. You can see the
most basic configuration in the last example.

\section{Dump output}
You can use the BIRD client to examine protocol behavior. Let us show some
actual output and walk through its meaning. This output was gathered by
connecting to a running BIRD with the configuration presented in this chapter.
The only difference was running an OSPF protocol to map the network topology.
Unrelated protocols were omitted from the output.

\begin{lstlisting}
birdc> dump protocols
  protocol mkernel1 state UP
    TABLE mroute4
    VIFs as in bitmaps:
      veth-br1 veth-r1 d1
    (S,G) entries in MFC in kernel:
      (10.3.3.1, 224.42.42.42, veth-br1) -> 111 110
\end{lstlisting}

\noindent First you can see the mkernel protocol running. It is connected to the default
table \ttt{mroute4}, with both filters set to accept all. Then mkernel lists
interfaces which were assigned an index for multicast routing. Follows a list
of (S,G) entries that were inserted in the kernel. Every entry has three
fields: a source address, a~group address and the incoming interface. After the
arrow there are two bitmaps from the original route, a set of incoming and
outgoing interfaces. Every interface from the list of VIFs has either a zero or
one, and they are in the same order.

\begin{lstlisting}[firstnumber=8]
  protocol igmp1 state UP
    TABLE mreq4
    Output filter: REJECT
    Interface d1 is up, querier
    Interface veth-r1 is up, other querier present
    Interface veth-br1 is up, querier
\end{lstlisting}

\noindent The IGMP dumps only its interfaces. An IGMP router can be either the querier on
the link, or just passively listening because other querier is present. That
happened on the interface veth-r1.

\begin{lstlisting}[firstnumber=14]
  protocol pim1 state UP
    TABLE master4
    Input filter: REJECT
    TABLE mreq4
    TABLE mroute4
    Output filter: REJECT
    Interface d1, up, holdtime 105, gen ID 413907797
    Interface veth-r1, up, holdtime 105, gen ID 957644869
      Neighbor 10.10.1.1 UP BIDIR
    Interface veth-br1, up, holdtime 105, gen ID 3449771873
      Neighbor 10.3.1.4 UP BIDIR
      Neighbor 10.3.1.2
    Group 224.42.42.42, RP 10.1.0.42,
      PIM joins: veth-br1; joined on: veth-br1
    Group 224.0.0.2, RP 10.1.0.42,
      PIM joins: veth-br1; joined on: veth-r1 veth-br1
    Group 224.0.0.5, RP 10.1.0.42,
      PIM joins: veth-br1; joined on: veth-r1 veth-br1
    Group 224.0.0.6, RP 10.1.0.42,
      PIM joins: veth-br1; joined on: veth-br1 veth-r1
    Group 224.0.0.13, RP 10.1.0.42,
      PIM joins: veth-br1; joined on: veth-r1 veth-br1
    Group 224.0.0.22, RP 10.1.0.42,
      PIM joins: veth-br1; joined on: veth-r1 veth-br1
    RP 10.1.0.42, RPF veth-r1
      iface veth-br1, DF state: Winner, DF: ::
      iface veth-r1, DF state: Lose, DF: 10.10.1.1
      iface d1, DF state: Winner, DF: ::
\end{lstlisting}

\noindent The PIM protocol has significantly more to output. First, PIM enabled
interfaces are listed with their settings and neighbors known there. You can
see from the flags that some neighbors are BIDIR-capable. There are packets coming
from the neighbor \ip{10.3.1.2}, but we haven't seen a hello from it yet. Then
there is a~list of groups with existing state. For every group, you can see its
RPA, interfaces which joined the group downstream by PIM Join mechanism, and
all interfaces for which a multicast request exists. Finally, for every RP tree
this router is connected to, there is a RP section. You can see the upstream as
the RPF, then the DF election state for every interface.

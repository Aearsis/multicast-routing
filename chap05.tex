\chapter{Multicast in the BIRD -- configuration}

To use the multicast routing features in the BIRD, you must enable at least two
protocols: PIM and mkernel. When any of the router's links have hosts
connected, you probably want to enable also the IGMP. This chapter covers their complete configuration.

\subsection{The IGMP}

The IGMP listens for hosts' multicast reports, and fills the request table in
the BIRD. The default values are well suited for the majority of use cases, but
you still need to include the IGMP configuraton section for IGMP to run. Do it
so whenever a router is connected to a link with local hosts. So, the most
common configuration will be as following:

\begin{lstlisting}
protocol igmp { }
\end{lstlisting}

Configurable variables have the same names as in the IGMP standard, \rfc{2246}.
Intervals are specified by time expressions to avoid unit problems. Have a look
at a configuration example:

\begin{lstlisting}
protocol igmp {
	interface "eth*" {
	  robustness 2;
	  query interval 125 s;
	  query response interval 10 s;
	  startup query count 2;
	  startup query interval 31 s;
	  last member query count 2;
	  last member query interval 1 s;
	};
	mreq4 { import all; };
}
\end{lstlisting}

All configuration variables are interface-specific. Let us go through their meaning:

\begin{description}[style=nextline]
\item[Robustness $k$]
  Loss of less than $k$ packets cannot cause protocol to have inconsistent
  states. Increasing $k$ will increase stability on lossy links, but groups
  will be left slower.

  Default value: 2

\item[Query interval $t$]
  A general query will be sent once per $t$. Decreasing $t$ will cause more
  control traffic, but faster convergence on lossy links.

  Default value: 125 seconds

\item[Query response interval $t$]
  Hosts have to send their report until $t$ passes. Larger values spread the
  burst of control traffic after a general query.

  Default value: 10 seconds

\item[Startup query count $c$, startup query interval $t$]
  When starting up, a loss of a query is critical, because it causes undetected
  delay before starting to froward traffic. First $c$ queries are sent in
  shorter interval $t$.

  Default values: 2 messages, 31 seconds

\item[Last member query count $c$, last member query interval $t$]
  After receiving a leave, $c$ group specific queries are sent to the group,
  once per $t$. After $c\cdot t$, router assumes there are no group members.
  Decreasing these will cause groups to be left faster.

  Default values: 2 messages, 1 second

\end{description}

There is also an implicit channel configuration for the multicast request
table. IGMP instance can have only one channel of type \texttt{mreq4}
configured. The defaults on channel are well suited here, so you can omit the
line 11 completely. The IGMP can never be exported a route, it rejects
everything.

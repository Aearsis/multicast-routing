\chapter{Protocol Independent Multicast}

Unicast routing protocols try to build a tree for every destination network
cooperatively. This tree is a spanning tree over some graph of networks and it
is rooted at its destination. We can call it the shortest-path tree with respect
to the destination. Every packet travels up the tree until it reaches its
destination. Alternatively, you can imagine them building a shortest-path
delivery tree for every source, rooted at the source, with traffic traveling
down the tree. This is what OSPF builds.

Multicast packets have no single destination, so multicast routing protocols
have to take a slightly different approach. For every multicast group, they still
try to build a tree connecting all traffic sources and receivers. Multicast
routing protocols often reuse the unicast routing topology to create the
multicast routing information base (MRIB).
\nomenclature{MRIB}{Multicast Routing Information Base}

Protocol Independent Multicast (PIM) is a whole family of multicast routing
protocols. Different protocols in this family are called PIM modes. The members of this
family share common mechanisms, but they differ in the routing logic. There are
situations where one mode can be better than the others.
\nomenclature{PIM}{Protocol Independent Multicast}

Every group managed by PIM has a \emph{Rendezvous-point} (RP). It is defined by
Rendezvous-point address (RPA) assigned by local configuration. The RP is a~router
specially configured for this purpose. Unicast delivery tree for the RPA is then
called a \emph{RP tree}. With respect to this tree, we can define a few more terms:
\nomenclature{RP}{Rendezvous-point}
\nomenclature{RPA}{Rendezvous-point address}

\begin{itemize}
\item\emph{Rendezvous-point link} (RPL) is simply a link where the RPA belongs.
  It is the root of the RP tree.
\item\emph{Reverse path forwarding interface} (RPF interface) with respect to
  an address is the next-hop interface. RPF neighbor is the next-hop router.
\item\emph{Upstream} is the direction towards the RPA, up the RP tree.
  \emph{Downstream} is the opposite direction.
\item\emph{Upstream interface} is then the RPF interface for the RPA.
\end{itemize}
\nomenclature{RPL}{Rendezvous-point link}
\nomenclature{RPF}{Reverse Path Forwarding}

While a unicast packet at first hops over a number of routers, then it is sent to
its destination, multicast packets are always sent on the link directly. This
way every router on the link can hear it. In order to avoid duplicate forwards, on
every link other than the RPL, election of a \emph{Designated Router} (DR)
takes place. This router is responsible for distributing
the information taken from IGMP messages to the DR on its upstream interface.
\nomenclature{DR}{Designated Router}

\pagebreak[4]

This information is distributed using PIM \emph{Join/Prune} messages. This
mechanism is almost a reimplementation of IGMPv2 in PIM context. When there is
a host on the link which joined a group $G$ by IGMP, or any router that joined~$G$ by
PIM, the DR joins $G$ on its upstream. Let us clarify the notation for joins and prunes:

\begin{itemize}
  \item join$(S, G)$ joins the source-specific tree for source $S$ and group $G$.
  \item join$(*, G)$ joins the shared RP tree for RPA$(G)$ and group $G$.
  \item join$(*, *, \hbox{RPA})$ joins the shared RP tree for a RPA given, effectively
    joining every group with this RPA.
\end{itemize}

\begin{figure}[htp]
\centering
\includegraphics[scale=1.00]{img/rp-tree.pdf}
\caption{An example RP tree}
\label{rp-tree}
\end{figure}

Let us first explain briefly how PIM Sparse Mode works. Suppose there is
already a RP tree built for a group $G$ with some receivers that joined~$(*,G)$.
One such network can be seen on figure \ref{rp-tree}.
Suddenly node $S$ starts sending multicast packets on a link. The DR on
this link ($A$, also called the first-hop router) takes these packets and
encapsulates them into ordinary unicast packets -- PIM Register messages. These
are then sent to the acting RP. The encapsulation is done by the routing
daemon. On BSD and Linux, the PIM daemon creates a~virtual interface, where the
kernel forwards unroutable multicast packets.

Incoming Register packets are decapsulated by the RP, which then temporarily
distributes them downstream. Routers are responsible for picking packets coming
on upstream and forwarding them down on links where they are acting as
designated routers according to the shared tree for $G$.

In the example, traffic between $S$ and $R$ flows from $S$ to $A$, then
encapsulated in register messages over $B$ to $RP$, then as multicast over $B$
and $D$ to $R$.

In the meantime, RP initiates a Join for $(S, G)$, which creates a distribution
tree for $G$ rooted at $S$. After all routers in between $A$ and RP joins the
source tree, packets will be forwarded from $A$ to RP naturally. At that time
RP starts to discard Register packets and it sends a Register-Stop message to
$A$. Encapsulating is then stopped and the traffic will flow as usual. So, now
the traffic follows the $(S,G)$ tree from $S$ to the RP, then the $(*,G)$ tree
down to all receivers.

\begin{figure}[htp]
\centering
\includegraphics[scale=1.00]{img/source-tree.pdf}
\caption{The same network with a $(S,G)$ tree}
\label{sg-tree}
\end{figure}

Eventually, for a receiver $R$, the DR on $R$'s link will send a $(S, G)$ join.
This join will propagate towards $S$ until reaching first router with $(S,G)$
state. At that time the receiver joins the source-specific tree for $(S,G)$ and
the packets use the shortest path from $S$ to $R$. The situation is then similar to
figure \ref{sg-tree}. Traffic uses the shortcut and it flows as multicast
directly from $S$ over $E$ and $F$ to $R$.

PIM Sparse Mode and PIM Dense Mode differ in their default downstream behavior.
While PIM-SM requires receivers to explicitly join, PIM-DM floods everything
over the RP tree by default. Branches without receivers are pruned afterwards.
The difference is that PIM-SM requires state information for every group with
receivers, while PIM-DM needs to hold state for every group with sources.
In these modes, the RP has to be an existing router and it must be running and have
enough processing power to decapsulate traffic from all sources until multicast
forwarding is established. Bidirectional PIM takes a different approach.
\nomenclature{PIM-SM}{PIM Sparse Mode}
\nomenclature{PIM-DM}{PIM Dense Mode}
\nomenclature{PIM-BIDIR}{PIM Bidirectional Mode}

\section{Bidirectional PIM}

Everything is made a bit simpler when we do not expect building specific trees
for every source. Routers can have a lot smaller routing tables and even build
them proactively. No encapsulation is then needed and processing power is
spared. Let us first clarify terms in the context of PIM-BIDIR.

Designated router is called the \emph{designated forwarder} (DF) and it takes additional
responsibilities. There are differences in how the election works. The RPA can be
just an arbitrary unicast-routable address and it is used just as the virtual root of the
distribution tree. It does not matter whether or not any router has the RPA
assigned. Every multicast packet from any source then travels up the tree,
being forwarded by the DFs, and then down to branches having receivers.
\nomenclature{DF}{Designated Forwarder}

\subsection{Designated forwarder election}

Every router running PIM-BIDIR knows from the MRIB what its upstream interface is,
and its metric towards RP. Router cannot be a DF on its upstream link, so
whenever it sends an offer, it advertises an infinite metric. Election never takes place
on the RPL, because it is the upstream link for every router.

There are four types of messages in the election: Offer, Winner, Backoff, and
Pass. Routers advertise their metrics as pairs of (protocol preference, distance).
The state machine with all details is described in \rfc{5015}. The main idea of the
election is that every router advertises its metric in Offer messages. When
a router hears a better Offer, it remains silent for a while. If it does not, after
advertising its metric three times, it claims itself a winner. Winning is
announced by Winner messages, and every other router notes the sender as the
winner and remembers its metric.

Two noteworthy events may happen. If the winner changes its metric to a~worse value,
it reannounces itself as a winner. When any other router has a~better metric,
it replies with an Offer immediately. Similarly, when a losing router changes its
metric to be better than the winner's, it sends an Offer.

When the metric changes at one router, it is often a reaction to some router losing its
link and it means that this change will propagate to other routers, changing their
metrics too. In order to spare bandwidth, the current winner reacts to a better offer
with a~Backoff message. It instructs the offering router to wait for a while until
unicast routing stabilizes. After a short period without other Offers, the winner
passes the DF role in a Pass message.

In the ideal conditions, there would be exactly one DF on every link. There are
situations, where no DF is elected, for example when all paths to the RPL are
lost. Then every node advertises an infinite metric, and no router is elected
winner. Also, when the DF fails silently, other nodes know only after few
missing hello messages. For the in-between period, no router is the DF.

Much worse would be if there were two DFs on a link, as that would lead to routing loops. The election protocol
is therefore biased to the safe side. For instance, an old DF sending a Pass immediately
stops being a DF, even if all Pass messages were to be lost. Even when they are
delivered, there is a short period without any router forwarding.

\subsection{Packet forwarding}

Suppose a source $S$ starts transmitting packets on its link $L_0$. Then
the currently acting DF will forward them immediately to its upstream link $L_1$.
The DF on $L_1$ forwards them to its upstream, $L_2$, and so on. This is repeated
until the packet reaches the RPL. As there is no DF on the RPL, forwarding upstream stops
there.

Simultaneously, all routers listen for packets on their upstream links, and they
forward packets from there to links where they are acting DFs. This way, a packet
is forwarded from $L_0, L_1, \dots$ in the reverse direction -- downstream.
Downstream forwarding is conditioned on having receivers active. Upstream
forwarding needs to be done even when there are no receivers.

You can see that every DF can forward packets for one group upstream and
downstream at the same time, explaining the name ``bidirectional''.

As opposed to other PIM modes, there is no need to encapsulate packets, thus
there is no different behavior on the first-hop router. Also, the RPA does not have
to be assigned to any router. This means that every router behaves the same, no
matter where in the tree it is, simplifying the overall protocol design.

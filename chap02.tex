\chapter{Protocol Independent Multicast}

Unicast routing protocols try to cooperatively build a tree for every
destination network. This tree is a spanning tree over some graph of networks,
and is rooted at its destination. Every packet travels up the tree until
reaching its destination.

Multicast packets have no single destination, so multicast routing protocols
have to take slightly different approach. For every multicast group they still
try to build a tree connecting all sources and listeners of traffic. Members of
Protocol Independent Multicast (PIM) routing protocols use multicast routing
information base (MRIB), which is often derived from the unicast routing table.

In this area BIRD can take advantage, because it is designed to run multiple
routing protocols at a time and share information between them. One can for
example route unicast traffic using BGP and use OSPF with different metric to
supply completely different topology to multicast routing.

Every group managed by PIM has its \emph{Rendezvous-point} address (RPA)
associated. Assigning it is a matter of local configuration. RP is a router
specially configured for this purpose. Unicast delivery tree for RPA is then
called RP tree. With respect to this tree we can define a few more terms:

\emph{Rendezvous-point link} is simply a link where RP address belongs.
\emph{Upstream interface} for every router is the next-hop interface towards
RP, and \emph{upstream neighbor} is the next-hop router.

While unicast packet at first hops over a number of routers, then it is sent to
its destination, multicast packets are always sent on the link directly. This
way every router on link can hear it. In order to avoid duplicate forwards, on
every link other than RPL election takes place. Winning router is called
\emph{designated router}. This designated router is responsible for
distribution of information taken from IGMP messages to upstream neighbor. This
is made by PIM Join/Prune messages. This mechanism is almost reimplementation
of IGMPv2 in PIM context. When there is a host on link which joined group $G$
by IGMP, or any router that joined $G$ by PIM, DR joins $G$ on its upstream.

Let me first explain briefly how PIM Sparse mode works. Suppose there already
is a tree built for group $G$ with some listeners. Suddenly node $S$ starts
sending multicast packets onto link. Then DR on this link ($A$) takes these
packets and encapsulates them into ordinary unicast packets. These are then
sent to RP with special type of PIM messages. This is done actively by the
routing daemon. RP decapsulates them, and temporarily distributes them down the
tree. Routers are responsible for picking packets coming on upstream and
forwarding them down on links where they are acting as designated routers.

In the meantime, $A$ communicates with RP through some routers in between. They
all establish a multicast routed path exclusively for this source. Then
encapsulating is stopped and the traffic can flow as usual. This complicated
reactive behaviour is needed, because building a distribution tree in advance
for every group and every possible source would be too expensive.

Furthermore, RP has to be an existing router, and must be running and have
enough processing power to decapsulate traffic from all sources until
muliticast forwarding is established. Bidirectional PIM takes different
approach.

\section{Bidirectional PIM}

Everything is made a lot simpler, when we do not expect hosts joining specific
sources. Routers can have a lot smaller routing tables, and even build them
proactively. No encapsulating is then needed and processing power is saved. So
first clarify things in terms of PIM-BIDIR.

Designated router is called designated forwarder (DF) and takes additional
responsibilities. There are differences in how the election works. RP can be
just unicast routable address, and is used just as virtual root of the
distribution tree. Every multicast packet from any source than travels up the
tree, being forwarded by DFs, and then down to branches having listeners.

\subsection{Designated forwarder election}

Every router running PIM-BIDIR knows from MRIB what its upstream interface is,
and its metric towards RP. Router cannot be DF on its upstream link, so
whenewer it sends offer, advertises infinite metric. Election never takes place
on RPL, because it is the upstream link for every router.

There are four types of messages in the election: Offer, Winner, Backoff and
Pass. Routers advertise their metrics as pairs of (protocol preference, distance).
State machine with all details is described in \rfc{5051}. The main idea of the
election is that every router advertises it's metric in Offer messages. When
a router hears better Offer, it remains silent for a while. If not, after
advertising its metric three times, it claims itself a winner. Winning is
announced by Winner messages, and every other router notes sender as the
winner and remembers its metric.

Two noteworthy events may happen. If winner changes its metric to worse value,
it reannounces itself as a winner. When any other router has better metric,
replies with an Offer immediately. Similarly, when losing router changes its
metric to be better than winner's, it sends an Offer.

When metric changes at one router, it is often reaction to some router losing
link, and means this change will propagate to other routers, changing their
metric too. In order to spare bandwidth, current winner reacts to better offer
with a Backoff message. It instructs target router to wait a while until
unicast routing stabilizes. After a short period without other Offers, winner
passes DF role in Pass message.

\TODO{Pathological situations}

\subsection{Packet forwarding}

Suppose source $S$ starts transmitting packets on its link $L_0$. Then
currently acting DF will forward them immediately to its upstream link $L_1$.
DF on $L_1$ forwards it to its upstream, $L_2$, and so on. This is repeated
until the packet reaches RPL. As there is no DF on RPL, forwarding up stops
there.

Simultaneously, all routers listen for packets on their upstream links, and
forwards packets from there to links where they are acting DFs. This way packet
is forwarded from $L_0, L_1, \dots$ in reverse direction -- downstream.
Downstream forwarding is conditioned by having listeners active. Upstream
forwarding needs to be done even when there are no listeners.

You can see that every DF can forward packets for one group upstream and
downstream at the same time, explaining the name "Bidirectional".

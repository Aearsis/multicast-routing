\chapter{Protocol Independent Multicast}

Unicast routing protocols try to cooperatively build a tree for every
destination network. This tree is a spanning tree over some graph of networks,
and is rooted at its destination. Every packet travels up the tree until it
reaching its destination.

Multicast packets have no single destination, so multicast routing protocols
have to take slightly different approach. For every multicast group they still
try to build a tree connecting all sources and listeners of traffic. Members of
Protocol Independent Multicast (PIM) routing protocols use multicast routing
information base (MRIB), which is often derived from the unicast routing table.

In this area BIRD can take advantage, because it is designed to run multiple
routing protocols at a time and share information between them. One can for
example route unicast traffic using BGP and use OSPF with different metric to
supply completely different topology to multicast routing.

Every group has its \emph{Rendezvous-point} address associated. Assigning it is
a matter of local configuration. RP address is simply an MRIB routable
address. MRIB is used to build a RP tree, where every router knows its upstream
(link towards RP). Network segment where RP address belongs is called RP link
(RPL).

Protocols from PIM family have some behaviour in common. On every link other
than RPL belongs election takes place. Winning router is called
\emph{designated router}. This designated router is responsible for
distribution of information taken from IGMP messages to other routers. This is
made by Join/Prune messages. This mechanism is almost reimplementation of
IGMPv2 in PIM context. When there is a host on link which joined group $G$ by
IGMP, or any router that joined $G$ by PIM, DR joins $G$ on
its upstream.

Let me first explain briefly how PIM Sparse mode works. Suppose there already
is a tree built for group $G$ with some listeners. Suddenly node $S$ starts
sending multicast packets onto link. Then DR on this link ($A$) takes these
packets and encapsulates them into ordinary unicast packets. These are then
sent to RP with special type of PIM messages. This is done actively by the
routing daemon. RP decapsulates them, and temporarily distributes them down the
tree. Routers are responsible for picking packets coming on upstream and
forwarding them down on links where they are designated routers.

In the meantime, $A$ communicates with RP through some routers in between. They
all establish a multicast routed path exclusively for this source. Then
encapsulating is stopped and the traffic can flow as usual. This complicated
reactive behaviour is needed, because building a distribution tree in advance
for every group and every possible source would be too expensive.

Furthermore, RP has to be an existing router, and must be running and have
enough processing power to decapsulate traffic from all sources until
muliticast forwarding is established. Bidirectional PIM takes different
approach.

\section{Bidirectional PIM}

Everything is made a lot simpler, when we do not expect hosts joining specific
sources. Routers can have a lot smaller routing tables, and even build them
proactively. No encapsulating is then needed and processing power is saved. So
first clarify things in terms of PIM-BIDIR.

Designated router is called designated forwarder (DF) and takes additional
responsibilities. There are differences in how the election works. RP can be
just unicast routable address, and is used just as virtual root of the
distribution tree. Every multicast packet from any source than travels up the
tree, being forwarded by DFs, and then down to branches having listeners.

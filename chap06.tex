\chapter{Testing}

An important part of development is testing. It is hard to test network
protocol, because a lot of code depends on the network state and temporary
conditions. Also, especially for testing routing protocols, you need multiple
networks, and run more than one instance of the routing daemon.

There are several tools that can virtualize network topology. Most of them are
based on virtual machines or containers. This is useful for configuration or
performance testing, but not that much for development. We've created
a lightweight testing framework based on Linux network namespaces to suit our
needs.

While developing the multicast routing features, we continuously tested it in
various testing environments, targeting the concrete feature being developed.
The IGMP implementation is tested agains Linux kernel, which works as the IGMP
host.

More complicated situation is with the PIM protocol, because we haven't
found any software solution doing the bidirectional variant of PIM. There are
routers from Cisco or Juniper which are able to manage a group in bidirectional
mode, but we haven't been provided any to test our implementation against it.
The network analyzer Wireshark is able to parse even Bidirectional PIM packets,
and we've examined the packets our implementation sends whether they are
constructed correctly. Of course, the PIM implementation in BIRDs works
correctly against itself.

\section{Netlab}
Our testing framework is called the \emph{netlab}. It is published in its
separate repository on Github\footnote{https://github.com/Aearsis/netlab}.

To use this framework, you must clone the repository into the root of the BIRD
repository, or change the path to BIRD inside the main executable of netlab.
Then, testing the BIRD, even in complex setups, is as simple as:

\begin{lstlisting}
$ sudo netlab/netlab
[netlab] # net_up
[netlab] # start
[r1] [r1n1] [r1n2] [r2] [r2n1] [r2n2] 
[netlab] # log r1
\end{lstlisting}

You can describe the testing network using a simple declarative language. For
example, a setup running two BIRDs connected directly by an Ethernet cable:

\begin{lstlisting}
NETLAB_CFG="$NETLAB_BASE/cfg"

netlab_node r1
netlab_node r2
if_veth r1 r2 10.10.1
\end{lstlisting}

It is very simple to add bridges, dummy nodes representing subnets, or to
create a namespace where to run other programs sending and receiving multicat
packets. This is the network we've done the most of the testing on:

\begin{figure}[htp]
\centering
\vskip 10cm
\caption{\TODO{The network scheme}}
\label{pim-channels}
\end{figure}

\section{Testing multicast}
To actually test the multicast forwarding, we have used simple python scripts.

\begin{lstlisting}[language=python]
#!/usr/bin/env python
import socket,struct,time

def setup(ga):
    sock = socket.socket(socket.AF_INET, socket.SOCK_DGRAM, socket.IPPROTO_UDP)
    sock.setsockopt(socket.SOL_SOCKET, socket.SO_REUSEADDR, 1)
    sock.setsockopt(socket.IPPROTO_IP, socket.IP_MULTICAST_TTL, 32)
    mreq = struct.pack("4sl", socket.inet_aton(ga), socket.INADDR_ANY)
    sock.setsockopt(socket.IPPROTO_IP, socket.IP_ADD_MEMBERSHIP, mreq)
    return sock

def send(ga, port, msg):
    sock = setup(ga)
    bmsg = msg.encode('us-ascii')
    while 1:
        sock.sendto(bmsg, (ga, port))
        time.sleep(0.1)

def recv(ga, port):
    sock = setup(ga)
    sock.bind((ga, port))
    while 1:
            data, addr = sock.recvfrom(1024)
            print(data.decode('us-ascii'))

# send.py
send('224.42.42.42', 42000, "What hath god wrought!");

# recv.py
recv('224.42.42.42', 42000)
\end{lstlisting}

For example, running the send script in the \ttt{send} namespace in the setup
above, the script joins a group and then starts multicasting a lot of messages.
If run a packet sniffer, for example \ttt{tcpdump}. you should see the traffic
travelling upstream, stopping at the RPL. When you run the receiving script in
the \ttt{recv} namespace, messages should start coming. You can use the netlab
for this purpose:

\begin{code}
[netlab] # shell send
[netlab send] # ./send.py
\end{code}

When testing, please be aware that PIM needs a path to the RP to work. In the
testing environment an OSPF instance is taking care of this, and it can take
a minute to initialize.

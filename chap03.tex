\chapter{BIRD Internet Routing Daemon}

BIRD is a routing daemon. It does not mean that the BIRD forward the traffic
itself, it just instructs the kernel how to do so. And how does the BIRD know?
It speaks with other routers using routing protocols. They share information
about the network topology and decide, which way should traffic take. Details
on how the information is shared and the path is calculated are up to the
specific protocol used. The BIRD currently\footnote{The latest released version
at the time of writing is 1.6.0} supports BGP, OSPF, RIP and Babel.

It is usualy enough to run only one routing protocol in your network, but there
can be use cases when you need more protocols running, for instance in a border
router between two network systems using different protocols already.

From the user point of view, the main units are the \emph{protocol} and the
\emph{routing table}. Routing tables are simply tables containing routes, the
same kind of tables as those in the operating system underneath the BIRD.
Routing tables in BIRD are kept only in its memory and the OS knows nothing
about them. You can have how many routing tables you want, and some of them do
not have to have corresponding table in the OS.

Protocols are instances of independent modules. Every protocol is connected
(typically) to one routing table. Protocols can add, modify and delete entries
in routing tables, and such changes are then announced to all connected
protocols. Different instances of one protocol, or even different protocols can
communicate through the table and exchange routing information in a language
both protocols understand.

\section{Filters}
Every protocol-table connection has two filters attached: the \emph{import} and
the \emph{export} filter. Every direction is from the table point of view, the
import one controls routes going from the protocol to the table, the exportone
controls routes going from the table to the protocol. Filter is called for
every route and decides whether the route should be accepted or rejected.

Default filters also show the most basic behavior. Import filter is by default
\texttt{all}, which means every route is accepted. Export filter is
\texttt{none} by default, causing every route to be rejected. This default
behavior is the reason why the BIRD does not magically work out of the box,
because running protocol instance have no knowledge of the network topology,
thus no information to share.

\section{Not-routing protocols}
Routing protocols often have some method to detect a neighbor failure using
some form of hello messages. But these often have long reaction period in the
order of tens of seconds. A protocol named BFD tries to provide faster, low
overhead failure detection. It serves as an advisor to other protocols.

Another protocol implemented in the BIRD is RAdv. It is used by routers to
advertise an available network prefix allowing hosts to autoconfigure.

\section{Special protocols}
Nothing from what we have discussed communicates with the kernel, yet protocols
and tables are all what the BIRD is made of. The synchronization of BIRDs
tables with the kernel routing table is made through a special protocol,
called, suprisingly, \texttt{kernel}. The kernel protocol behaves like an usual
protocol, and is connected to one table through some filters. Every route it
receives propagates to the kernel, and every route from the kernel is announced
back. On systems where there can be more than one routing table (e.g. Linux),
you have to run a separate instance of the kernel protocol for each. The
scanning is done periodically. Do not forget to set the export filter for
kernel protocol to something more sensible than \texttt{none}.

Another necessary protocol is the \texttt{device} protocol. It does not send
nor receive route updates, but periodically scans available network interfaces.
Unless you have a very specific setup, this protocol is needed because other
protocols need to know what devices do they have under control.

Three more special protocols are implemented. The \texttt{static} protocol adds
static routes. \texttt{Pipe} connects two tables and copies routes between
them. Of course, this is useful only with filters, otherwise you could have
just used only one table. Last, the \texttt{direct} protocol generates routes
that directly belong to some interface. That is useful, when your operating
system cannot track route origin and would allow the BIRD to permanently
overwrite these routes. This is the case for BSD, on Linux the BIRD cannot
change routes from other sources.

\section{IP agnosticity}
The creators of the BIRD have made a decision, that the BIRD will be able to
route IPv4 and IPv6, but not at the same time. The choice is made during
compilation, and a bunch of macro definitions will compile the BIRD either for
IPv4, or for IPv6. If you have a dual-stack network, you have to run both BIRDs
at the same time, and configure them separately.

This choice is good because of performance and memory reasons. Internal
structures can be optimized for the address length, which means they can be
narly four times faster and lighter for IPv4. However, nowadays it is usual to
have tunnels mixing both internet protocols together, and a routing daemon must
be able to route IPv6 traffic in a tunnel over IPv4 network. That is one of the
reasons the BIRD team have chosen to make a big change and generalize internal
structures to run both versions at the same time.

My work is based on this dual-stack integrated version. Moreover, it is
following an experimental development branch, where protocol-table connections
are generalized and called \texttt{channels}. A protocol can have any number of
channels of any type, thus receive route updates from more tables. PIM heavily
uses this, because it is connected to three tables all the time.

\section{Configuration}
Because of channels and IP agnosticity, configuration differs between the
latest released version of BIRD and the version my work is based on. As I would
like to give configuration samples in the following chapters, it wouldn't make
much sense to give them for a different version. But this configuration files
will not work in the officially released BIRD 1.6.0.

Let us have a look at some simple configuration file:
\lstinputlisting{cfg/simple.cfg}

This is the most basic configuration actually doing something. We can see the
definition of a kernel protocol, scanning kernel routing table every 20 seconds
for new routes, and exporting every route from BIRDs main table to the kernel.
Next to it is the device protocol, set to scan every 10 seconds. Then we're
running RIP, filtering no route out. That means RIP will get all routes read
from kernel, exchange them with its neighbors, and put new routes into the main
table. It will communicate on any interface given. Everything for IPv4 only.
Let's try something more interesting, omitting the kernel and device protocol:

\lstinputlisting{cfg/rip_ospf.cfg}

\TODO{Check if these are working :)}

Using natural language, these directives are very easily readable and
self-explanatory. There are multiple levels of messages in the log, and
especially the level \texttt{trace} has categories; \texttt{debug packets}
will turn on debug messages about packets for the RIP protocol. The RIP will
run only on interfaces with name starting with "eth", and its routes will be
taken into account only when having RIP metric under 10.

OSPF configuration defines the backbone area, spanning over the device's wlan
interfaces. The static protocol defines some routes, we can see an usual route
having next hop, blackhole route, which means every packet chosing this route
will get silently dropped, and a recursive route, where the next hop is the
same as the next hop for IP \ip{192.168.0.42}.

\chapter{BIRD Internet Routing Daemon}

BIRD is a routing daemon. It does not mean that the BIRD forwards the traffic
itself, it just instructs the kernel how to do so. And how does the BIRD know?
It speaks with other routers using routing protocols. They share information
about the network topology and decide, which way should traffic take. Details
on how the information is shared and the path is calculated are up to the
specific protocol used. The BIRD currently\footnote{The latest released version
at the time of writing is 1.6.0.} supports BGP, OSPF, RIP and Babel.
\nomenclature{BGP}{Border Gateway Protocol}
\nomenclature{OSPF}{Open Shortest Path First}
\nomenclature{BGP}{Routing Information Protocol}

It is usualy enough to run only one routing protocol in your network, but there
can be use cases when you need more protocols running simultaneously, for instance in a~border
router between two network systems already using different protocols.

From the user point of view, the main units are the \emph{protocol} and the
\emph{routing table}. Routing tables are simply tables containing routes, the
same kind of tables as those in the operating system underneath the BIRD.
Routing tables in BIRD are kept only in its memory and the OS knows nothing
about them. You can have as many routing tables as you want, and some of them do
not have to have a~corresponding table in the OS.

Protocols are instances of independent modules. Every protocol is connected
(typically) to a single routing table. Protocols can add, modify and delete entries
in routing tables, and such changes are then announced to all connected
protocols. Different instances of one protocol, or even different protocols can
communicate through the table and exchange routing information in a language
undestood by both protocols.

\section{Filters}
Every protocol-table connection has two \emph{filters} attached: the
\emph{import} and the \emph{export} filter. They are named from the point of
view of a table. The import filter controls routes going from the protocol to
the table, the export filter controls routes going from the table to the
protocol. The filter is called for every route and it decides whether the route
should be accepted or rejected.

Default filters also show the most basic behavior. The import filter is by default
\texttt{all}, which means every route is accepted. The export filter is
\texttt{none} by default, causing every route to be rejected. This default
behavior is the reason why the BIRD does not magically work out of the box,
because running protocol instance have no knowledge of the network topology,
thus no information to share.

Before accepting the route, filters may also modify the route's attributes.
This mechanism is targeted to, but not limited to, setting route's metrics
while transfering the route between different protocols. But you can do almost
anything with the route, including, e.g., changing the next hop.

\section{Non-routing protocols}
Routing protocols often have some method to detect a neighbor failure using
some form of hello messages. But these often have long reaction period in the
order of tens of seconds. A protocol named BFD tries to provide faster, low
overhead failure detection. It serves as an advisor to other protocols.
\nomenclature{BFD}{Bidirectional Forwarding Detection}

Another protocol implemented in the BIRD is RAdv. It is used by IPV6 routers to
advertise an available network prefix allowing hosts to autoconfigure.
\nomenclature{RAdv}{Router Advertisement}

\section{Special protocols}
Nothing from what we have discussed communicates with the kernel, yet protocols
and tables are all what the BIRD is made of. The synchronization of BIRD's
tables with the kernel routing table is made through a special protocol
called, suprisingly, \texttt{kernel}. The kernel protocol behaves like an usual
protocol, and it is connected to one table through some filters. Every route it
receives propagates to the kernel, and every route from the kernel is announced
back. On systems where there can be more than one routing table (e.g., Linux),
you have to run a separate instance of the kernel protocol for each. The
scanning is done periodically. Do not forget to set the export filter for
kernel protocol to something more sensible than \texttt{none}.

Another necessary protocol is the \texttt{device} protocol. It does not send
nor receive route updates, but periodically scans available network interfaces.
Unless you have a very specific setup, this protocol is needed because other
protocols need to know what devices do they have under control.

Three more special protocols are implemented. You can specify some routes in
the configuration, then the \texttt{static} protocol will add them to the BIRD's
tables. The \texttt{pipe} connects two tables and it copies routes between
them. Of course, this is useful only with filters, otherwise you could have
just used only one table. Finally, the \texttt{direct} protocol generates routes
that directly belong to some interface. That is useful when your operating
system cannot track route origin and it would allow the BIRD to permanently
overwrite these routes. This is the case for BSD, on Linux the BIRD cannot
change routes from other sources.

\section{IP agnosticity}
When BIRD was initially designed, its creators decided that the it will be able to
route both IPv4 and IPv6, but not at the same time. The choice is made during
compilation and a bunch of macro definitions will compile the BIRD either for
IPv4, or for IPv6. If you have a dual-stack network, you have to run both BIRDs
at the same time and configure them separately.

This choice is good because of performance and memory reasons. Internal
structures can be optimized for the given address length, which means they can be
nearly four times faster and lighter for IPv4. However, nowadays it is usual to
have tunnels mixing both internet protocols together and a routing daemon must
be able to route IPv6 traffic in a tunnel over IPv4 network. That is one of the
reasons the BIRD team have chosen to make a big change and generalize internal
structures to run both versions at the same time.

My work is based on this dual-stack integrated version. Moreover, it is
following an experimental development branch, where protocol-table connections
are generalized and called \texttt{channels}. A protocol can have any number of
channels of any type, thus receive route updates from more tables. PIM heavily
uses this, because it is connected to three tables all the time.

Having two IP versions in one daemon introduces differences between routing
tables. Every routing table has a type attached, and channels can usualy work
only with specific table types. The default routing table for the type
\ttt{ipv4} is the table \ttt{master4}. The notation will be important in the
configuration.

\section{Configuration}
Because of channels and IP agnosticity, configuration differs between the
latest released version of BIRD and the version my work is based on. As I would
like to give configuration samples in the following chapters, it wouldn't make
much sense to give them for a different version. But this configuration files
will not work in the officially released BIRD 1.6.0.

Let us have a look at a simple configuration file:

\lstinputlisting{cfg/simple.cfg}

\noindent This is the most basic configuration actually doing something. We can see the
definition of a kernel protocol, scanning kernel routing table every 20 seconds
for new routes. The only channel \ttt{ipv4} of the kernel protocol have the
export filter set to \ttt{all}, exporting every route from BIRD's routing table
\ttt{master4} to the kernel. Next to it is the device protocol, set to scan
every 10 seconds. Then we run the RIP protocol, filtering no route out. That
means RIP will get all routes read
from kernel, exchange them with its neighbors, and put new routes into the main
table. It will communicate on all available interfaces.

\clearpage

\noindent Let's try something more interesting, omitting the kernel and device protocol:

\lstinputlisting{cfg/rip_ospf.cfg}

\noindent Using natural language, these directives are easily readable and
self-explanatory. There are multiple levels of messages in the log, and
especially the level \texttt{trace} has categories; \texttt{debug packets}
will turn on debug messages about packets for the RIP protocol. The RIP will
run only on interfaces with name starting with~``eth'', and its routes will be
taken into account only when having RIP metric under 10.

OSPF configuration defines the backbone area, spanning over the device's wlan
interfaces. The static protocol defines some routes, we can see an usual route
having next hop, a blackhole route, which means every packet chosing this route
will get silently dropped, and a recursive route, where the next hop is the
same as the next hop for IP \ip{192.168.0.42}.

\section{The internals}
\label{bird-internals}
The overall architecture can be called event-driven. When you implement a new
protocol, you register hooks to be called when there is an interface status
change, route update, protocol is about to be shut down, reconfigured, and so on.
The most interesting hooks can be registered on sockets and timers.

BIRD sockets are a thin layer over traditional Unix sockets. They try to hide
incompatibilities between diferent Unixes by providing system-dependent
implementation. But the main benefit of them is simplified receiving and
sending packets using hooks.

What might be suprising, BIRD is single-threaded. It has only one IO loop that
dispatches these hooks. That means you do not have to mess with locking in the
common code, but it comes with a few drawbacks. In the loop, there is
a \texttt{select} syscall waiting on all sockets. Whenever there are some data
waiting, BIRD calls a \verb|rx_hook| on that socket.

Another very common usage pattern in network protocols are timers. You often
need to send a message every ten seconds, or let something timeout. It would be
too expensive to have three threads sleeping for every network prefix, so BIRD
introduces internal timers. Timers have one hook and a time when it should be
called. BIRD does not promise an exact time, but it ensures the hook would not
be called sooner. The timeout for the select in the loop is set to the time
until the first timer should be fired.

When doing some computations that can take a long time, for example going
through a complete routing table, you cannot do it in single hook as there can be
other, possibly time sensitive, protocols waiting for their time. A mechanism
called events was made for this purpose. An event can be \emph{scheduled},
and its hook will run at the end of current IO loop cycle. It is recommended to
split such long-running pieces of code into several event calls, simulating
multiple threads running with less overhead than actually having them.

Because there are computations in the IO loop that can take longer time, it is
not possible to have precise timers. For that reason, internal timers had the
precision of one second. When a timer is delayed by two seconds, it does not
matter much when it was set to ten seconds. But the same delay introduced to
a few-millisecond timer can be critical.

Traversing over a routing table spread over more iterations of the main IO loop
brings several problems. When any event is called inbetween, you have to deal
with added entries, removed entries (especially the current one being removed),
and complete rehashes of the table. Routing table is stored in clever hash
table called FIB, \emph{Forwarding Information Base}. This table follows
a special property making elements keep their relative order when the table is
rehashed. Furthermore, every node keeps a list of iterators, and when the node
is removed, all its iterators are moved to the next element.

\subsection{Resources}
Common software problem, dangeorus especially for daemons, is memory leakage.
When writing network software, one often needs to allocate temporary structures
for holding additional information tied to an interface, neighbor or a network
prefix. It can happen and often happens that you lose all handles to allocated
memory and you cannot deallocate it. This forgotten memory cummulates over time
and memory usage of your daemon grows. When it overgrows the limited memory
in embedded devices, your daemon is killed.

BIRD tries to fight (not only) this problem with dedicated mechanisms for
memory allocations. There are several memory pools from which resources are
allocated. You can then deallocate either the resources one by one, or the
whole pool at once. For example, every protocol instance has its own pool,
which is freed when the instance is shut down.

This mechanism has more advantages. Every resource type can have its destructor
specified, so it can remove itself from lists, free related memory and so. To
give an example, timers has their own resource type, and when a timer is freed,
it removes itself from the list of upcoming timers. Or, when a socket is
closed, it deallocates its buffers.

Resource pools offer multiple memory allocation strategies, making them more
efficient. BIRD offers memory blocks (the usual strategy similar to
malloc/free), linear memory pool and Slabs. Linear memory pools are handy when
creating a temporary structure (which will be freed as a whole after using).
Slabs are efficient in managing a lot of fixed-size blocks, e.g., the FIB nodes.

\subsection{System dependent parts}
A routing daemon, whether we want or not, does have to communicate tightly with
the kernel. The BIRD is written with all Unix-like operating systems in mind,
and it was originally developed at least for Linux and BSD. However, the
development of last years is strongly focused on Linux only.

To avoid having a lot of compile-time checks and conditionally compiled blocks,
BIRD has a source code subtree, which is linked according to the target
operating system. This subtree is called \ttt{sysdep}. There are parts which
are special for Linux or BSD, but also parts common for all Unixes. The
creators of BIRD wanted to make porting to different operating systems as
simple as possible.

When you look in the sysdep code, you will find for example functions to read
system time, to communicate with the kernel or the socket abstraction layer,
but also the whole implemetation of the main IO loop. If you look into the
forbidden lands of the kernel protocol, you will find arcane magic.

\section{The user interface}
To communicate with the running daemon, the BIRD opens a control socket, and offers
a well-defined protocol which you can use to control the running daemon. To
make it actually usable, the BIRD is provided with a shell-like utility,
\ttt{birdc}. You can run commands to examine the routing tables, dump protocol
information or to change the configuration on-the-fly by disabling and enabling
protocol instances. The client needs to open a connection to the control
socket, and therefore you usually have to run it with the root privileges.

When you update the configuration, it is possible to reload it to the running
daemon without losing routing tables in the memory. BIRD tries to avoid
restarting as much as possible. There are mechanisms how to reconfigure
a running protocol, how to restart it without losing its routes, or how to
restart only those protocols that changed too much.

\chapter{Multicast in IPv4}

IP multicast was added as an additional addressing model to unicast. First
references can be found in \rfc{966}. This RFC defines multicast traffic
with many responsibilities for routers (called "multicast agents" in that time)
and was soon obsoleted. Latest specification can be found in \rfc{1112}.

In local networks, multicasts are similar to broadcasts. Every packet is to be
transmitted to every host on that link, and host can decide according to
destination address whether or not receive and process it.

Differences are notable on internetwork level. While broadcasts should be
either forwarded to all links (or filtered completely) by router, multicasts
should have more controled behavior.

\section{Multicast group}

First term we need to understand is multicast group. It is represented by an IP
address allocated from dedicated region \prefix{224.0.0.0/4}. It is further
divided (as per \rfc{5771}) to:

\begin{tabular}{lll}
	224.0.0&/24	& Local Network Control Block \\
	224.0.1&/24	& Internetwork Control Block \\
	224.2&/16	& SDP/SAP Block \\
	232&/8		& Source-Specific Multicast Block \\
	233&/8		& GLOP Block \\
	239&/8		& Administratively Scoped Block \\
\end{tabular}

And to AD-HOC blocks in between.

Host membership in group is fully optional. Neither number of members in group,
nor number of groups host belongs to, is limited and can be even zero. Host
does not have to belong to group to send packets into it. Packets are
distributed in the same "best effort" manner as unicasts -- packets can be
delivered to all, some or no hosts, in any order.

There are three levels of conformance with IP multicast. Level 0 means no
conformance. Level 1 requires host to be able to send packets to multicast
groups. This level still requires little code in networking stack, because
sending packet to multicast group is essentialy the same as sending it as
unicast packet with destination set to the group address.

Level 2 means fully conformant, and requires implementation of the host part of
Internet Group Management Protocol (IGMP).

\section{Internet Group Management Protocol}

Current version of IGMP is 3, and is described in \rfc{3376}. However, all
versions are backward compatible. All nodes in one network falls back to the
least version all of them support.

Purpose of IGMP is to exchange information between hosts and routers about
group membership. Note that it is mandatory to implement IGMP to be part of
multicast group, but until multicasts cross boundary of one network, it is
delivered as broadcast to all hosts.

Let's start with IGMP version 1 specified in Appendix I. of \rfc{1112}. There
are just two types of messages: membership query and membership report.

Membership query is sent periodically by routers, and announces to hosts there
is IGMPv1 capable router. This query is sent to group \ip{224.0.0.1} with TTL 1, and
therefore is local network limited.

Every host is ought to be member of this group, hence the common name
ALL-HOSTS. When hosts receives query, replies with membership report in random
delay up to 10 seconds. No host should send a report for this group.

Report is sent to the group being reported. As noted above, host does not have
to report to listen. Filtering multicast traffic is made by other means.
Sending a report declares there is someone interested in this group on this
particular link.

When, during the random delay, host receives a report, it stops his own timer.
This reduces the amount of control traffic.

When host decides to leave a group, it does nothing. If it was the only host in
that group, on the next query no one will report and routers will stop
forwarding.

IGMPv2 \cite{rfc2236} comes with some major changes. First, router can send query to one
specific group. That means it can keep timer for every group separately, and
asking for reports doess not generate unnecessary trafic in every other group.

Second, the maximum delay between query and report (previously fixed 10
seconds) is now specified in each query, and is configurable. Network
administrator can tune this variable for his needs. By turning it down, hosts
will reply faster and routers can faster stop forwarding. Giving longer delays
can improve stability on very slow or unreliable links.

The most notable change though is new message type - leave group. When host
sends a report, it sets a flag. Other hosts without flag remain silent. When
flagged host leaves a group, it sends the leave group message. Other hosts
react to orher host's leave in similar way to query.

When router hears leave, it starts a timer. This timer can be stopped by
receiving report. Expiring this timer immediately changes this group state to
not having any members. This way less unnecessary traffic is forwarded.

IGMPv3 comes with additional fields and more complex protocol logic. It is
dedicated to source-specific multicast forwarding and includes messages to keep
set of sources hosts on current link want to receive.

For purpose of this thesis, only IGMPv2 was implemented, as PIM-BIDIR is not
source specific.

\TODO More about backward compatibility?




\TODO Eth filter
\TODO scoping
\TODO mutlicast on host level

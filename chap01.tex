\chapter{Multicast in IPv4}

IP multicast was added as an additional addressing model. First
references can be found in \rfc{966}. This RFC defines multicast traffic
with many responsibilities for routers (called "multicast agents" in that time)
and was soon obsoleted. Latest specification can be found in \rfc{1112}.

In local networks, multicasts are similar to broadcasts. Every packet is to be
transmitted to every host on that link, and host can decide according to
destination address whether or not to receive and process or drop it. In other
words, you need neither a multicast routing capable router, nor any special
network hardware to use multicast in one network.

Differences are notable on internetwork level. While broadcasts should be
either forwarded to all links or filtered completely by routers, multicasts
should have more controled behavior.

\section{Multicast groups}

First term we need to understand is multicast group. It is represented by an IP
address allocated from dedicated region \prefix{224.0.0.0/4}. It is further
divided into blocks, which have additional semantics. From routing point of
view, only the Local Network Control Block (\prefix{224.0.0.0/24} is
interesting. Multicasts in this group must never be forwarded and must never
cross the boundary of local network. This block is often used for unicast
routing protocols communication.

Host membership in group is fully optional. Neither number of members in group,
nor number of groups host belongs to, is limited and can be even zero. Host
does not have to belong to group to send packets into it. Packets are
distributed in the same "best effort" manner as unicast ones -- can be
delivered to all, some or no hosts, and in any order.

There are three levels of conformance with IP multicast. Level 0 means no
conformance. Level 1 requires host to be able to send packets to multicast
groups. This level still requires little little code in networking stack, because
sending packet to multicast group is very similar to sending it as unicast
packet with destination set to the group address. While it doesn't make much
sense to send unicast packets destinated outside local network prefix directly
to link, it is the expected behavior with multicast.

Level 2 means fully conformant, able to send and receive multicast traffic.

\section{Internet Group Management Protocol}

As already said, membership in multicast group is optional. By default, there
is only one group that every host listens to - named ALL-HOSTS with address
\ip{224.0.0.1}. There must be way for hosts to express group membership.
A protocol was set for this purpose, and is called IGMP.

Current version of IGMP is 3, and is described in \rfc{3376}. However, all
versions are backward compatible. All nodes in one network fall back to the
least version that all of them support.

Purpose of IGMP is to exchange information between hosts and routers about
group membership. Note that it is mandatory to implement IGMP to be part of
multicast group, but until multicast packet crosses boundary of one network, it
is delivered as broadcast to all hosts. As such all hosts receive it without
the routers knowledge.

Let us start with IGMP version 1 specified in Appendix I. of \rfc{1112}. There
are just two types of messages: membership query and membership report.
Membership query is sent periodically by routers, and announces to hosts there
is IGMPv1 capable router. This query is sent to ALL-HOSTS group with TTL 1, and
therefore is local network limited. When hosts receives query, it replies with
a membership report in random delay up to 10 seconds. No host should send
reports for ALL-HOSTS group, membership there is implicit and mandatory.

Report is sent to the group being reported. As noted above, host does not have
to report to listen. Filtering multicast traffic is made by other means.
Sending a report declares there is someone interested in this group on this
particular link. When, during the random delay, host receives a report, it
stops his own timer and remain silent. This reduces the amount of control
traffic. When host decides to leave a group, it does nothing. If it was the
only host in that group, on the next query no one will report and routers will
stop forwarding.

IGMPv2 \cite{rfc2236} comes with some major changes. First, router can send
query to one specific group. That means it can keep timer for every group
separately, and asking for reports does not generate unnecessary trafic in
every other group. Second, the maximum delay between query and report
(previously fixed 10 seconds) is now specified in each query, and is
configurable. Network administrator can tune this variable for his needs. By
turning it down, hosts will reply faster and routers can faster stop
forwarding. Giving longer delays can improve stability on very slow or
unreliable links.

The most notable change though is new message type - leave group. When host
sends a report, it sets a flag. Other hosts without flag remain silent. When
flagged host leaves a group, it sends the leave group message. Other hosts
react to orher hosts leave in similar way to query. When router hears leave,
it starts a timer. This timer can be stopped by receiving report. Expiring this
timer immediately changes this group state to not having any members. This way
less unnecessary traffic is forwarded.

IGMPv3 comes with additional fields and more complex protocol logic. It is
dedicated to source-specific multicast forwarding and includes messages to keep
set of sources hosts on current link want to receive.

For purpose of this thesis, only IGMPv2 was implemented, as PIM-BIDIR is not
source specific routing protocol.

\section{Link layer}

Even though multicast is specified on network layer, it needs some level of
support from link layer. Let me explain how multicast traffic integrates with
the most common combination of IP over ethernet.

First problem we need to solve is which MAC address to use for multicast group
addresses. When sending unicast packet, ARP is used to determine MAC address.
As multicast is technically a broadcast on link layer, it would be possible to
use broadcast address \mac{FF:FF:FF:FF:FF:FF}.
But that would mean every multicast packet on network must be received by all
hosts and examined by operating systems, because there is no way how to filter
it on hardware level. For this purpose, every ethernet address that starts with
1 (that means, the least significant bit of the first byte) is considered to be
broadcast address and is to be flooded to all nodes on this network. Network
hardware generally do not make difference between them, so we can associate
specific meaning to diferent MAC addresses.

For IPv4 multicast, region of MAC addresses was allocated. This region starts
at \mac{01:00:5E:00:00:00} and ends with \mac{01:00:5E:7F:FF:FF}. To construct
MAC address from multicast group address, one puts lower 23 bits of IP address
to lower bits of the MAC address. This way, network interfaces can implement
rough filter, that can drop frames host knows nothing about. Additional
filtering is still needed because of those missing bits.

Multicast router needs to receive packets for all groups, because IGMP reports
are sent there. To enable this behavior NICs often have "all multicast" mode,
which lets al multicast frames pass through. Or we can put the NIC to
promiscuous mode, which passes all frames to the operating system.

\subsection{IGMP snooping}

What was written above is true according to specifications, but not much
reflecting the real state of art. Link layer hardware such as switches often do
make a difference and use a technique called IGMP snooping. This is described
in \rfc{4541}. This RFC is only informational, a lot of network hardware with
this functionality was produced before this RFC and do not behave in any
standardized way. Network switch performing IGMP snooping understands IGMP and
listens to router communicating with hosts, and filters multicast traffic on
the link layer.

At the first glance this seems reasonable, because it stops forwarding packets
to network segments without any listeners, thus saving bandwidth. On the other
hand, this high level functionality is expected from network layer, and
reimplementing it in link layer brings many corner cases, which can prevent
local multicast from working at all. Usually a combination of network without
multicast router with IGMP snooping switch can cause unexpected troubles.

\section{Kernel implementation}

Kernel interface for multicast is pretty complicated. At least differences
between UNIXes are not big. Let us start with code to receive multicast trafic
in an ordinary application. It is as simple as opening a socket and setting it
it up by some setsockopt calls:

\TODO{Example -- code to open socket, set up multicast, join group, ...}

\noindent One thing to note here. When there is socket opened which joined group, the host
itself joins the group by means of IGMP. And when host is joined, every other
multicast enabled socket will get packets addressed to that group. So, you
still need to filter those at application level.

When router daemon wants to receive all IGMP packets for all groups, it
would have to join every group possible. That is certainly not the way to go.
IGMPv2 suggests using IP option Router Alert. Every IP packet with this option
set should be passed to sockets which orders it by this setsockopt call:

\TODO{Example -- code calling setsockopt IP\_ROUTER\_ALERT}

\noindent There are multiple problems with this. Backward compatibility with
IGMPv1 is broken, and standard does not force hosts to do so -- and some (e.g.
old Cisco) routers do not do it.

To my surprise, there is only one way how to get all IGMP packets coming on
some interface. That is opening the IGMP control socket.

\subsection{IGMP control socket}

When we want to control kernel multicast routing tables, we need to open
control socket. That is, raw IP socket with type \texttt{IPPROTO\_IGMP}. Then,
sequence of setsockopt calls is needed:

\TODO{Example -- socket, MRT\_INIT, MRT\_PIM}

There can be only one socket on which MRT\_INIT succeeds. This is the one
control socket, through which we can control multicast routing tables in a way
explained below. This part could be relatively easily rewritten (at kernel
side) to some sane netlink api.

But, there is more -- this is the only socket on a host, which receives all
IGMP packets. So, it is not even theoretically possible to run more than one
multicast routing daemon on one system.\footnote{Unless isolating them in
different network namespaces, more about it later.}

\chapter*{Introduction}
\addcontentsline{toc}{chapter}{Introduction}

% short intro to uni/broad/multicast

There are multiple ways how to use a computer network to communicate between two
or more machines. In the most common technology, based on Internet Protocol
(IP), one can send a message (called packet) from one machine, destined to
another one. That is usually called unicast and is by far the most well-known
method. Or one can send a packet to all machines in some network -- that is
called broadcast. However, there are applications where both of these
are suboptimal. A good example can be conference telephony, games or video
streaming.

When public television company broadcasts final match at Olympic games, it can
have millions of receivers. The company does not want to send every packet in
millions of copies by unicast. However, it does not want to flood every
computer connected to The Internet by broadcast either. That is why multicast have
been invented. Multicast is a general way how to transmit traffic from group of
sources to group of listeners.

In order to deliver packets between networks, the network itself has to do some
computations. One of them is routing -- collaboratively finding optimal path
for packet, for some definition of optimal.
World of unicast routing is well inhabited. There are lots of unicast routing
protocols with completely different approaches. When it comes to routing
multicast, much is different.

Multicast itself was added as an extension to IPv4 and requires special
handling in network hardware. When just one router on a way does not support
it, multicast will not work. As such it suffered from a lack of support for a long
time. And because this support was often broken, it is still not widely used.

Supporting multicast on the side of the receiver is easy, and one can receive
multicast in all current major operating systems. For example in UNIXes, it is
a matter of few syscalls. But routing multicast traffic is a different thing
-- one needs to run a routing daemon that communicates with hosts and other
routers, speaks at least two different network protocols and controls the
kernel's routing tables. There already are such daemons that are traditionally
used: e.g. mrouted and pimd.

The purpose of this thesis is to explain how multicast works, introduce the reader to a world of multicast routing and implement one of existing multicast routing
protocols. It would make no sense to implement another standalone routing
daemon.

There once was a student project, that grew to worldwide usage -- The BIRD
internet routing daemon. It currently supports a handful of unicast routing
protocols at a time. There are use-cases in which having closer communication
between unicast and multicast routing can be advantage. Let the primary goal of
this thesis be to teach BIRD multicast routing.

The protocol chosen for this thesis is PIM-BIDIR. It is somehow simplified
protocol from the PIM family because it doesn't need to hold state information
based on traffic source. That makes it less demanding on network hardware.
